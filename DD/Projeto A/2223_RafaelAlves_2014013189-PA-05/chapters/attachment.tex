\section{Trabalho desenvolvido da Meta 1}
O trabalho desenvolvido para a Meta 1 foi uma pesquisa simples para o tema proposto pelo professor Luís Santos. 

A nível de arquitetura do trabalho em Latex foi o básico. Eu já tinha trabalhado com Latex no ano letivo anterior (muito pouco) e fiquei a gostar mas soube a pouco. No verão de 2021-2022 decidi aprender mais sobre Latex e criei o meu próprio repositório GitHub, \textit{First Conctact With Latex}. Fiquei a conhecer melhor os \textit{packages}, a implementação e como se tratava os ficheiros para os capítulos, a bibliografia, entre outros. Contudo para esta meta e trabalho inicial já possuía noções básicas da implementação e funcionamento de um trabalho em Latex. Foi enviado com alguns erros e foi feito mais trabalho de pesquisa para mitigar os erros para a meta seguinte.


\section{Trabalho desenvolvido da Meta 2}
Para a Meta 2 foi preciso uma pesquisa mais profunda, exigiu um estudo profundo para perceber com detalhe vários assuntos da proposta para esta meta. 

As implementações que efetuei foi: criação de ficheiros individuais para cada capítulo, implementação da bibliografia, introdução e melhorar alguns erros como páginas em branco, folhas com erros de formatação. 

Para a próxima meta tenciono explorar e desenvolver melhor o último capítulo, Arquitetura de Aplicações Web e acrescentar a este trabalho a página de Acrónimos.

Apesar de todo o trabalho exigente em Latex, tem sido motivador pesquisar cada vez mais para conseguir entregar a tempo e horas cada meta.

\section{Trabalho desenvolvido da Meta 4}
Esta meta exigiu um grande trabalho com uma carga horária que ocupou bastante tempo da semana. Apesar do grande trabalho que foi preciso eu já tinha trabalhado com cluster em linux e já tinha conhecimento da experiência que ia ter. Valeu a pena mais uma semana de aprendizagem e esforço para completar esta meta!
Entrego o relatório mas está por finalizar, falta quase para eu conseguir o que quero entregar. Na próxima meta irá completo.

\section{Trabalho desenvolvido da Meta 5}
Para concluir esta meta foi preciso consolidar muitos conhecimentos práticos para a implementação das experiências. Exigiu bastantes horas para a implementações do estudo com as máquinas virtuais e ir compreendendo os vários erros em alguns serviços. Neta meta final apesar do esforço para a implementar, entrego também a meta 4 concluída. Com mais algum tempo gostava de ter explorado por exemplo o Galera Cluster e o Cluster Control. Tenho noção que existe tópicos incompletos e outros que devia ter implementado mas foi um Projeto A muito exigente e conjugar com o resto das cadeiras nao é facil.