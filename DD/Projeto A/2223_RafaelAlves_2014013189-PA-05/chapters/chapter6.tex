\section{MySQL}
O MySQL é a \ac{BD} de código aberto (\textit{Open Source}) mais popular do mundo que utiliza a linguagem \ac{SQL} como interface. De acordo com a DB-Engines, o MySQL é a segunda base de dados mais popular, atrás da Base de Dados Oracle.
MySQL é um \ac{SGBD} relacional que suporta vários motores de armazenamento (InnoDB, \ac{CSV}, entre outros) e dependendo do seu ambiente de programação, pode ser introduzido SQL diretamente (por exemplo, para gerar relatórios), incorporar instruções \ac{SQL} em código escrito noutra linguagem, ou utilizar uma \ac{API} específica da linguagem que esconde a sintaxe \ac{SQL}.

\section{MariaDB}
O MariaDB é um dos mais populares servidores de bases de dados relacionais do mundo. Foi feito pelos criadores do MySQL e também é de código aberto (\textit{Open Source}). A intenção é também manter uma elevada compatibilidade com MySQL, assegurando uma equivalência de bibliotecas e correspondência exata com \ac{API} e comandos MySQL.
Em versões mais recentes o MariaDB inclui o Galera Cluster o que facilita bastante o processo de criação de um cluster usando o \ac{SGBD} MariaDB.